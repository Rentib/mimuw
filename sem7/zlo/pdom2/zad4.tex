\documentclass[11pt]{article}

\usepackage[british]{babel}
\usepackage[T1]{fontenc}
\usepackage[utf8]{inputenc}
\usepackage{mathtools}                       % matma
\usepackage{amsfonts,amsmath,amssymb,amsthm} % matma
\usepackage{braket}
\usepackage{tikz}
\usepackage{hyperref}

\newcommand{\dd}{\mathinner{{\ldotp}{\ldotp}}}

\title{Computational Complexity, homework problem 4, Fall 2024}
\author{Stanisław Bitner}
\date{\today}

\begin{document}
\maketitle

\section*{Problem 4}
The solution is \textbf{heavily} based on the theorem $14.12$ of \href{https://parameterized-algorithms.mimuw.edu.pl/parameterized-algorithms.pdf}{book on moodle}.\\
The reduction is from 3-colouring.\\
We are given a graph $G=\langle V, E \rangle, N=|V|$, let $k = \lceil \frac{N}{\log_3{\sqrt{\log_2{N}}}} \rceil$.\\
Arbitrarily partition vertices into $k$ parts $V_1,V_2,\ldots,V_k$,\\
$\forall_{i \in 1 \dd k} |V_i| \le \frac{N}{k} = \log_3{\sqrt{\log_2{N}}}$.\\
For every part $V_i$, list all possible good $3-$colourings of $G[V_i]$.\\
There are at most $3^{|V_i|} = 3^{\log_3{\sqrt{\log_2{N}}}} = \sqrt{\log_2{N}}$ such colourings.\\
Note that for sufficiently big $N$: $3^{|V_i|} = \sqrt{\log_2{N}} \le \frac{N}{\log_3{\sqrt{\log_2{N}}}} = k $.\\
If for any part $V_i$ there is no $3$-colouring, then there is no $3$-colouring for the whole graph.\\
If there is less than $\sqrt{\log_2{N}}$ colourings for any part, then copy the last one to fill the gap.\\
Now we have a set of exactly $\sqrt{\log_2{N}}$ colourings for each part.\\
Let $c_i^1, c_i^2, \ldots, c_i^{\sqrt{\log_2{N}}}$ be the colourings for part $V_i$.\\
Now we create a new graph with vertices $[1 \dd k] \times [1 \dd k]$.\\
Now we add edges $(j,i),(j',i')$ if and only if $i \neq i'$,\\
$j \le \sqrt{\log_2{N}}, j' \le \sqrt{\log_2{N}}$ and $c_i^j \cup c_{i'}^{j'}$ is a
valid $3$-colouring of $G[V_i \cup V_{i'}]$.\\
Then if in $G'$ there is a $k$-clique, then there exists a $3$-colouring of $G$.\\
Note that the number of edges incident to a single vertex in $G'$ is at most
$\sqrt{\log_2{N}}^2 = \log_2{N}$ as well as for every $1\le b \le k$ there is no
edge between vertices $\{ (a,b) \big| 1 \le a \le k \}$. Thus, the graph $G'$
fulfils the requirements of the graph from the problem statement. Hence, if the
homework problem was solvable in $\mathcal{O}(c^n)$ or here $\mathcal{O}(c^k)$,
the ETH would be falsified.\\
The reduction is obviously polynomial. 
The intricacies of why finding a clique in $G'$ is equivalent to finding a
$3$-colouring of $G$ are explained in depth in the theorem $14.12$ of the book,
so I will not repeat them here.

\end{document}
