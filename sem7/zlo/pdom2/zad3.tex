\documentclass[11pt]{article}

\usepackage[english]{babel}
\usepackage[T1]{fontenc}
\usepackage[utf8]{inputenc}
\usepackage{mathtools}                       % matma
\usepackage{amsfonts,amsmath,amssymb,amsthm} % matma
\usepackage{braket}
\usepackage{tikz}
\usepackage{esvect}

\title{Computational Complexity, homework problem 3, Fall 2024}
\author{Stanisław Bitner}
\date{\today}

\begin{document}
\maketitle

\section*{Problem 3}
Let's define two morphisms $\varphi_1, \varphi_2 : \left\{ 0,1 \right\}
\rightarrow \left\{ 0,1 \right\}^*$:
\begin{align*}
    &\varphi_1(1) = 000, & &\varphi_2(1) = 111,\\
    &\varphi_1(0) = 011, & &\varphi_2(0) = 001,\\
    &\vv{a} \in \left\{0,1\right\}^*\\
    &\varphi_1(\vv{a}) = \varphi_1(a_1) \ldots \varphi_1(a_{|a|})\\
    &\varphi_2(\vv{a}) = \varphi_2(a_1) \ldots \varphi_2(a_{|a|}).
\end{align*}\\
Let $A',B' \subseteq \left\{0,1\right\}^{d'}$. From the lecture, we know that
solving orthogonal vectors problem on those two sets falsifies SETH.\\
Observe that for any two vectors $\vv{a} \in A', \vv{b} \in B'$
$$ \vv{a} \cdot \vv{b} = 0 \iff h(\varphi_1(\vv{a}), \varphi_2(\vv{b})) \le d', $$
where $h$ denotes the hamming distance between two vectors.
\\
Thus, one may solve the orthogonal vectors problem by solving the problem
described in the homework, which proves that solving it in
$\mathcal{O}(d^{100} \cdot n^{2-\delta})$, where $\delta>0$, falsifies SETH. \qed\\
The input for the homework statement would be:\\
$d:3d', k:d', A:\{\varphi_1(\vv{a}) : \vv{a} \in A'\}, B:\{\varphi_2(\vv{b}) : \vv{b} \in B'\}$.\\
What's more the transformation is obviously linear.

\end{document}
