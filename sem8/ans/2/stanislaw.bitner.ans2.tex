\documentclass[11pt]{article}

\usepackage[polish]{babel}
\usepackage[T1]{fontenc}
\usepackage[utf8]{inputenc}
\usepackage{mathtools}                       % matma
\usepackage{amsfonts,amsmath,amssymb,amsthm} % matma
\usepackage{braket}
\usepackage{tikz}
\usepackage[margin=1in]{geometry}

\newcommand{\dd}{\mathinner{{\ldotp}{\ldotp}}}
\newcommand{\quasiLinear}[1]{\widetilde{\mathcal{O}}\left(#1\right)}

\title{Algorytmy Najkrótszych ścieżek -- Praca Domowa 2}
\author{Stanisław Bitner}
\date{\today}

\begin{document}
\maketitle

\section{}

Niech k=$\delta_G(u,v)$. Rozważmy najkrótszą ścieżkę w grafie G z $u$ do $v$.
$P = p_1 p_2 \dd p_k$, $p_1=u, p_k=v$. Niech $z=p_{k-1}$.

W grafie $G^2$ można dostać się z $u$ do $v$ na kilka sposobów. Idąc po
wierzchołkach ze ścieżki $P$ jedną z najkrótszych ścieżek jest $p_1 p_3 \dd
p_{k-1} p_k$ lub $p_1 p_3 \dd p_{k-2} p_{k}$, taka ścieżka ma długość
$\lceil\frac{k-1}{2}\rceil$.

Przypuśćmy, że istnieje ścieżka $P'=p'_1 p'_2 \dd p'_{k'}$ taka, że $p'_1 =
p_1$, $p'_{k'}=p_k$ oraz $|P'| = k'-1 < \lceil\frac{k-1}{2}\rceil$. To oznacza,
że w szczególności w grafie $G$ istniała ścieżka z $u$ do $v$ długości $d$
taka, że $k'-1 \le d \le 2(k'-1) < 2\lceil\frac{k-1}{2}\rceil \le k$, $d<k$,
ale $d\ge \delta_G(u,v)=k-1$, zatem $d \ge k-1$, czyli ścieżka $P'$ przechodzi po
wierzchołkach z jednej z najkrótszych ścieżek między $u$ i $v$ w grafie $G$,
albo po wierzchołkach ze ścieżki o $1$ dłuższej niż najkrótsza, zatem nie może
być krótsza niż $\lceil\frac{k-1}{2}\rceil$, sprzeczność.

\subsection*{(a)}

Długość najkrótszej ścieżki $u \rightsquigarrow v$ w $G$ jest nieparzysta, więc
długość najkrótszej ścieżki $u \rightsquigarrow v$ w $G^2$ jest równa
$\frac{k}{2}$. Zauważmy, że aby dojść do sąsiada $v$ można przejść tą samą
ścieżką $p_1 p_3 \dd p_{k-1}$ i potem jeśli $y=p_{k-1}$, to kończymy, a jeśli
nie, to $y$ jest połączone jedną krawędzią z $p_{k-1}$, więc $\delta_{G^2}(u,y)
\in \{\frac{k}{2}-1, \frac{k}{2}\}$, zatem $\delta_{G^2}(u,y) \le
\delta_{G^2}(u,v)$. Dla $z=p_{k-1}$ $\delta_{G^2}(u,z)=\delta_{G^2}(u,v)-1$.

\subsection*{(b)}

Długość najkrótszej ścieżki $u \rightsquigarrow v$ w $G$ jest parzysta, więc
długość najkrótszej ścieżki $u \rightsquigarrow v$ w $G^2$ jest równa
$\frac{k-1}{2}$. Zauważmy, że aby dojść do sąsiada $v$ można przejść tą samą
ścieżką $p_1 p_3 \dd p_{k-2}$ i potem jeśli $y=p_{k-1}$, to przechodzimy tam w
jednym kroku, a jeśli nie, to możemy przejść w dwóch krokach, więc
$\delta_{G^2}(u,y) \in \{\frac{k-1}{2}, \frac{k+1}{2}\}$, zatem
$\delta_{G^2}(u,y) \ge \delta_{G^2}(u,v)$.

\section{}
Rozwiązanie będzie randomizowane.\\
Z każdego wierzchołka w grafie prowadzimy pętelkę do niego samego o wadze $0$,
to powoduje, że ścieżki długości dokładnie $k$ będą uwzględniały też ścieżki
krótsze.\\
Tworzymy $\log{n}$ hitting setów $H_0,H_1,\dd,H_{\lceil\log_{n}\rceil}$, $|H_i|
= 42\frac{n}{2^i}\log{n}$. Najkrótsze ścieżki długości $k$ z dużym
prawdopodobieństwem mają któryś wierzchołek w $H_{\lceil\log{k}\rceil}$
(najkrótszych ścieżek różnych długości jest $\mathcal{O}(n^3)$). Dla każdego
$H_i$ liczymy algorytmem Bellmana-Forda najkrótsze ścieżki w grafie $G$ oraz
grafie transponowanym $G^T$ długości co najwyżej $2^i$, robimy to w czasie
$\mathcal{O}(\frac{n}{2^i}\log{n} \cdot 2^i m) = \mathcal{O}(nm\log{n})$.
Robimy to dla każdego z logarytmicznie wielu hitting setów, więc dostajemy
złożoność $\mathcal{O}(nm\log{n} \cdot \log{n}) = \tilde{\mathcal{O}}(nm)$.\\
Aby odpowiedzieć na zapytanie o $\delta_G^k(s,t)$ wyznaczamy najpierw $i$,
takie że $2^i \ge k$, następnie dla każdego wierzchołka z $H_i$ sprawdzamy
wszystkie możliwe ścieżki:
$$
\delta_G^k(s,t) = \min \{ \delta_{G^T}^l(v,s) + \delta_G^{k-l}(v,t) \big| v \in H_i, l=0 \dd k \}
$$
Wyliczenie takiego minimum zajmuje $\mathcal{O}(\frac{n}{2^i}\log{n} \cdot k)
\le \mathcal{O}(\frac{n}{k} \log{n} \cdot k) = \mathcal{O}(n\log{n}) =
\tilde{\mathcal{O}}(n)$.

\section{}

Na początku w czasie $\mathcal{O}(n+m)$ znajdujemy najkrótszą ścieżkę $s
\rightsquigarrow t$, $P = e_1 \dd e_k$. Robimy to dynamikiem, tak jak na
ćwiczeniach. Dla każdej krawędzi $e$ spoza $P$, $\delta_{G-e}(s,t) =
\delta_{G}(s,t)$. Obliczenie wartości dla krawędzi ze ścieżki $P$ jest
trudniejsze.

Na początek wyliczamy $\delta_{G}(s,v)$ oraz $\delta_{G^T}(t,v)$ dla każdego $v
\in G$, gdzie $G^T$, to transponowany graf $G$. Wyniki tych obliczeń trzymamy w
jakichś tablicach. Potem tworzymy dwie kopie krawędzi $E_u$ i $E_v$. $E_u$
sortujemy po kolejności topologicznej wejściowego wierzchołka krawędzi, $E_v$
po kolejności topologicznej wyjściowego wierzchołka krawędzi. Tworzymy
strukturę danych trzymającą krawędzi $uv$ oraz wartości $\delta_{G}(u,v) +
w(uv) + \delta_{G^T}(t,u)$ pozwalającą na logarytmiczne dodawanie krawędzi,
usuwanie krawędzi oraz znajdywanie minimalnej wartości (wystarczy jakiś set z
cpp). Dodatkowo obliczamy tablicę $ts(v) = \text{numer w kolejności
topologicznej\;}v$.

Przechodzimy po kolejnych krawędziach spośród $e_1 \dd e_k$. Rozpatrując
krawędź $e_i=uv$ do struktury dodajemy krawędzie $u'v'$ takie, że $ts(u') <
ts(u)$ oraz wyrzucamy wszystkie krawędzie $u'v'$ takie, że
$ts(v') < ts(v)$. Robimy to w zamortyzowanym czasie $\mathcal{O}(m
\log m)$, za pomocą dwóch wskaźników na tablicach $E_u$ i $E_v$. Po tych
operacjach struktura zawiera wszystkie krawędzie przechodzące „dookoła''
krawędzi $uv$. Następnie kopiujemy wartości tablic $\delta_{G}(s,w),
\delta_{G^T}(t,w)$ dla każdego $w$ takiego, że $ts(u) < ts(w) <
ts(v)$, Potem je zerujemy i obliczamy nie uwzględniając krawędzi $e_i$ (jak w
zwykłym dynamiku do obliczania odległości w DAGu).
\[
\delta_{G-e_i}(s,t) =
\min \left\{
    \begin{array}{l}
        \displaystyle \min \big\{ \delta_{G}(s,u') + w(u'v') + \delta_{G^T}(t,v') \;\big|\; ts(u') < ts(u) \land ts(v) < ts(v') \big\}, \\[1.2em]
        \displaystyle \min \big\{ \delta_{G}(s,w) + \delta_{G^T}(t,w) \;\big|\; ts(u) < ts(w) < ts(v) \big\}
    \end{array}
\right.
\]
Pierwszy element obliczamy w $\mathcal{O}(\log m)$ wykonując zapytanie do
struktury, drugi obliczamy w $\mathcal{O}(|\{ w \big| ts(u) < ts(w) < ts(v)\}|)$.
Po obliczeniu przywracamy skopiowane wartości $\delta_G$ i $\delta_{G^T}$.
Nietrudno zauważyć, że żaden wierzchołek $w$ się nie powtórzy, gdyż gdyby się
powtórzył, to graf nie byłby DAGiem. To natomiast oznacza cały algorytm działa
w zamortyzowanym czasie $\quasiLinear{n+m}$.

\end{document}
