\documentclass[12pt, a4paper]{article}
\usepackage{fancyhdr}
\usepackage[T1]{fontenc}
\usepackage[T1]{polski}
\usepackage[polish]{babel}
\usepackage[utf8]{inputenc}
\usepackage{microtype}
\usepackage{setspace}
\usepackage{fancyhdr}
\usepackage{lipsum}
\usepackage{amsmath}
\usepackage{amsthm}
\usepackage{amsfonts}
\usepackage{enumerate}
\usepackage[margin=1in]{geometry} % zmniejsza margines
\usepackage{nopageno} % usuwa numery stron

\title{\textit{
    \textbf{BSK - Aplikacje Webowe}\\
}}
\author{Stanisław Bitner - 438247}
\date{\today}

\begin{document}
\maketitle

\section*{Wstęp}
Był udostępniony kod źródłowy, więc zacząłem od znalezienia wszystkich flag za
pomocą rekurencyjnego grepa.
Okazało się, że flagi są w następujących miejscach:
\begin{itemize}
  \item w pliku \texttt{/flag.txt};
  \item w stopce admina;
  \item w mikroserwisie.
\end{itemize}

\section*{/flag.txt}
Szablon używany do pokazywania flagi nie jest sprawdzany w kodzie źródłowym,
więc wystarczyło podać plik z flagą jako szablon wysłanej kartki.

\section*{Stopka}
Treść wysyłanych kartek jest sprawdzana jedynie po stronie klienta, więc można
było wysłać zapytania bez użycia przeglądarki i dzięki temu zrobić
HTMLInjection. Łatwo było więc wysłać kartkę do admina, która zmusza go do
wysłania kartki do nas. Aby zdobyć CSRF token admina można było zrobić GET
podstrony /create, w której jest on jawnie zapisany przy użyciu szablonu
Django. Pozostała jeszcze kwestia zmuszenia admina do odczytania naszej kartki,
ale okazało się, że xss-bot robi to automatycznie, więc nie trzeba było nic
z tym robić.

\section*{Mikroserwis}
Mikroserwis jest dostępny jedynie poprzez localhosta, więc trzeba było się do
niego jakoś dobrać. Znalezienie adresu mikroserwisu było proste po przeczytaniu
kodu. Każda kartka ma opcje DownloadPNG, która pobiera zawartość kartki.
Zawartość kartki można oczywiście podmienić robiąc przekierowanie na adres
mikroserwisu. Wystarczyło wysłać do nas samych kartkę z przekierowaniem
i później wykonać dla niej DownloadPNG.

\section*{Treść Flag}
\begin{itemize}
  \item /flag.txt:   FLAG\{JeszczeJednaFlagaZaPunkty\}
  \item Stopka:      FLAG\{ToJestFlagaZeStopki\}
  \item Mikroserwis: FLAG\{71a4b4fd2214b808e4942dfb06c717878399a04c\}
\end{itemize}

\end{document}
